\documentclass{report}
\usepackage[a4paper,margin=1in,footskip=0.25in]{geometry}

\input{index}


\begin{document}

\title{Proofs problems}
\author{Joshua Morton}
\maketitle

\chapter{Intuitive Proofs}


\begin{fact}{The pigeonhole principle}{PHP}
  \textbf{Simple form:} If $n + 1$ objects are placed into $n$ boxes, then at least one box has at least two objects in it. \\
  \textbf{General form:} If $kn + 1$ objects are placed into $n$ boxes, then at least one box has at least $k + 1$ objects in it.
\end{fact}

\begin{proposition*}{}
  If one chooses $n+1$ numbers from $\{1, 2, 3, \ldots, 2n\}$, it is guaranteed that two of the numbers they chose are consecutive.
\end{proposition*}

\begin{proof}
  TODO
\end{proof}

\begin{proposition*}{}
  If one selects any $n + 1$ numbers from the set $\{1, 2,\ldots,2n\}$, then two of the selected numbers will sum to $2n + 1.$
\end{proposition*}

\begin{proof}
  TODO
\end{proof}

\begin{proposition*}{}
  If one chooses 31 numbers from the set $\{1,2,3,\ldots,60\}$, then two of the numbers must be relatively prime.
\end{proposition*}

\begin{proof}
  TODO
\end{proof}

\begin{problem*}{}
  Determine whether or not the pigeonhole principle guarantees that two students at your school have the same 3-letter initals.
\end{problem*}

TODO

\chapter{Direct proofs}


\end{document}