\documentclass{report}
\usepackage[a4paper,margin=1in,footskip=0.25in]{geometry}

\input{index}


\begin{document}

\title{Proofs problems}
\author{Joshua Morton}
\maketitle

\chapter{Intuitive Proofs}


\begin{fact}{The pigeonhole principle}{PHP}
  \textbf{Simple form:} If $n + 1$ objects are placed into $n$ boxes, then at least one box has at least two objects in it. \\
  \textbf{General form:} If $kn + 1$ objects are placed into $n$ boxes, then at least one box has at least $k + 1$ objects in it.
\end{fact}

\begin{proposition*}{}
  If one chooses $n+1$ numbers from $\{1, 2, 3, \ldots, 2n\}$, it is guaranteed that two of the numbers they chose are consecutive.
\end{proposition*}

\begin{proof}
  TODO
\end{proof}

\begin{proposition*}{}
  If one selects any $n + 1$ numbers from the set $\{1, 2,\ldots,2n\}$, then two of the selected numbers will sum to $2n + 1.$
\end{proposition*}

\begin{proof}
  TODO
\end{proof}

\begin{proposition*}{}
  If one chooses 31 numbers from the set $\{1,2,3,\ldots,60\}$, then two of the numbers must be relatively prime.
\end{proposition*}

\begin{proof}
  TODO
\end{proof}

\begin{problem*}{}
  Determine whether or not the pigeonhole principle guarantees that two students at your school have the same 3-letter initals.
\end{problem*}

TODO

\chapter{Direct proofs}

\begin{fact}{}{IntegerClosure}
  The sum of integers in an integer, the difference of integers is an integer, and the product of integers is an integer.
\end{fact}

\begin{definition}{Even and odd integers}{EvenAndOdd}
  \begin{itemize}
    \item[$\bullet$] An integer $n$ is even if $n = 2k$ for some integer $k$;
    \item[$\bullet$] An integer $n$ is odd if $n = 2k + 1$ for some integer $k$.
  \end{itemize}

  Fact: Any integer is either even or odd.

\end{definition}

\begin{proposition*}{}
  The sum of an even integer and an odd integer is odd.
\end{proposition*}

\begin{proof}
  Assume that $n$ is an even integer and that $m$ is an odd integer.
  By the definition of even and odd numbers $n = 2a$ and $m = 2b + 1$ for some integers $a$ and $b$.
  Then, $$n + m = (2a) + (2b + 1) = 2a + 2b + 1 = 2(a + b) + 1.$$
  And since $a + b$ is an integer by Fact 2.0.1, we have shown that $n + m = 2k + 1$ where $k = a + b$.
  Therefore by the definition of an odd integer this means that $a + b$ is odd.
\end{proof}

\begin{proposition*}{}
  The product of two even integers is even.
\end{proposition*}

\begin{proof}
  Assume that $n$ and $m$ are even integers.
  By the definition of an even integer $n = 2a$ and $m = 2b$ for some integers $a$ and $b$.
  Then, $$nm = (2a)(2b) = 4ab = 2(2ab).$$
  And since $2ab$ is an integer by Fact 2.0.1, we have shown that $nm = 2k$ where $k = 2ab$.
  Therefore by the definition of an even integer this means that $nm$ is even.
\end{proof}

\begin{proposition*}{}
  The product of two odd integers is odd.
\end{proposition*}

\begin{proof}
  Assume that $n$ and $m$ are odd integers.
  By the definition of an odd integer this means that $n = 2a + 1$ and $m = 2b + 1$ for some integers $a$ and $b$.
  Then, $$nm = (2a + 1)(2b + 1) = 4ab + 2a + 2b + 1 = 2(2ab + a + b) + 1.$$
  And since $2ab + a + b$ is an integer by Fact 2.0.1, we have shown that $nm = 2k + 1$ where $k = 2ab + a + b$.
  Therefore by the definition of an odd integer this means that $nm$ is odd.
\end{proof}

\begin{proposition*}{}
  The product of an even integer and an odd integer is even.
\end{proposition*}

\begin{proof}
  Assume that $n$ is an even integer and $m$ is an odd integer.
  By the definition of an even and odd integer this means that $n = 2a$ and $m = 2b + 1$ for some integers $a$ and $b$.
  Then, $$nm = (2a)(2b + 1) = 4ab + 2a = 2(2ab + a).$$
  Since $2ab + a$ is an integer by Fact 2.0.1, we have shown that $nm = 2k$ where $k = 2ab + a$.
  Therefore by the definition of an even integer this means that $nm$ is even.
\end{proof}

\begin{proposition*}{}
  An even integer squared is an even integer.
\end{proposition*}

\begin{proof}
  Assume that $n$ is an even integer.
  By the definition of an even integer $n = 2a$ for some integer $a$.
  Then, $$n^2 = (2a)^2 = 4a^2 = 2(2a^2).$$
  Since $2a^2$ is an integer by Fact 2.0.1, we have shown that $n^2 = 2k$ where $k = 2a^2$.
  Therefore by the definition of an even integer this means that $n^2$ is even.
\end{proof}

\begin{definition}{}{}
  A nonzero integer $a$ is said to \emph{divide} an integer $b$ if $b = ak$ for some integer $k$.
  When $a$ does divide $b$, we write $``a \mid b"$ and when $a$ does not divide $b$ we write $``a \nmid b."$
\end{definition}

\begin{proposition}{}{}
  If $d \mid a$ and $d \mid b$ then $d \mid a + b$.
\end{proposition}

\begin{proof}
  Assume that $d \mid a$ and $d \mid b$.
  By the definition of divisibility $a = dk$ and $b = dl$ for some integers $k$ and $l$.
  Then, $$a + b = dk + dl = d(k + l).$$
  Since $k + l$ is an integer by Fact 2.0.1, we have shown that $a + b = dq$ where $q = k + l$.
  Therefore by the definition of divisibility this means that $d \mid a + b$.
\end{proof}

\begin{proposition}{}{}
  If $d \mid b$ then $d \mid -b$.
\end{proposition}

\begin{proof}
  Assume that $d \mid b$.
  By the definition of divisibility $dk = b$ for some integer $k$.
  Then, $$-b = -(dk) = d(-k).$$
  Since $-k$ is an integer by Fact 2.0.1, we have shown that $-b = dq$ where $q = -k$.
  Therefore by the definition of divisibility this means that $d \mid -b$.
\end{proof}

\begin{proposition}{}{}
  If $d \mid b$ then $-d \mid b$.
\end{proposition}

\begin{proof}
  Assume that $d \mid b$.
  By the definition of divisibility $dk = b$ for some integer $k$.
  Then, $$b = dk = --dk = -d(-k)$$
  Since $-k$ is an integer by Fact 2.0.1, we have shown that $b = -dq$ where $q = -k$.
  Therefore by the definition of divisibility this means that $-d \mid b$.
\end{proof}

\end{document}