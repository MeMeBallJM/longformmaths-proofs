\documentclass{report}
\usepackage[a4paper,margin=1in,footskip=0.25in]{geometry}


% ams packages (usefull for a bunch of stuff)
\usepackage{amsmath}
\usepackage{amsthm}
\usepackage{amssymb}

% tikz (creates drawings)
\usepackage{tikz}
\usetikzlibrary{arrows.meta}

% tcolorbox (draws colored boxes)
\usepackage[most,many,breakable]{tcolorbox}

% Add colors
\usepackage{xcolor}

% Links
\usepackage{hyperref}

% Variable with objects (tbh I don't really know)
\usepackage{varwidth}


\usepackage[framemethod=TikZ]{mdframed}



% Bold vectors instead of arrow
\let\vec\mathbf

% Change real and imaginary components to look normal
\renewcommand{\Re}{\operatorname{Re}}
\renewcommand{\Im}{\operatorname{Im}}

% Change QED symbol
% \renewcommand\qedsymbol{\emph{Do you believe me now!?}}
\renewcommand\qedsymbol{\emph{Quick maths}}

% Blackboard bold letters
\newcommand{\A}{\mathbb{A}}
\newcommand{\C}{\mathbb{C}}
\newcommand{\F}{\mathbb{F}}
\newcommand{\K}{\mathbb{K}}
\newcommand{\N}{\mathbb{N}}
\newcommand{\Q}{\mathbb{Q}}
\newcommand{\R}{\mathbb{R}}
\newcommand{\V}{\mathbb{V}}
\newcommand{\Z}{\mathbb{Z}}


% Theorem environment
\definecolor{theorem_color_background}{HTML}{F8E7EE}
\definecolor{theorem_color_frame}{HTML}{C73A71}
\definecolor{theorem_color_title}{HTML}{FFFFFF}
\newtcbtheorem[number within=section]{theorem}{Theorem}
{
  skin=enhanced,
  boxrule = 0.75mm,
  sharp corners,
  % Title settings
  attach boxed title to top left = {
      yshift*=-\tcboxedtitleheight/2,
      xshift = 10mm,
    },
  boxed title style = {colback=theorem_color_frame, sharp corners},
  % Adjust colors
  colback= theorem_color_background,
  colframe= theorem_color_frame,
  coltitle = theorem_color_title,
}{theorem}

% Proposition environment
\definecolor{proposition_color_background}{HTML}{FCF6EB}
\definecolor{proposition_color_frame}{HTML}{DCA323}
\definecolor{proposition_color_title}{HTML}{000000}
\newtcbtheorem[number within=section]{proposition}{Proposition}
{
  skin=enhanced,
  boxrule = 0.75mm,
  sharp corners,
  % Title settings
  attach boxed title to top left = {
      yshift*=-\tcboxedtitleheight/2,
      xshift = 10mm,
    },
  boxed title style = {colback=proposition_color_frame, sharp corners},
  % Adjust colors
  colback= proposition_color_background,
  colframe= proposition_color_frame,
  coltitle = proposition_color_title,
}{proposition}

% Lemma environment
\definecolor{lemma_color_background}{HTML}{DFF0DC}
\definecolor{lemma_color_frame}{HTML}{468C3B}
\definecolor{lemma_color_title}{HTML}{FFFFFF}
\newtcbtheorem[number within=section]{lemma}{Lemma}
{
  skin=enhanced,
  boxrule = 0.75mm,
  sharp corners,
  % Title settings
  attach boxed title to top left = {
      yshift*=-\tcboxedtitleheight/2,
      xshift = 10mm,
    },
  boxed title style = {colback=lemma_color_frame, sharp corners},
  % Adjust colors
  colback= lemma_color_background,
  colframe= lemma_color_frame,
  coltitle = lemma_color_title,
}{lemma}


% Fact environment
\definecolor{fact_color_background}{HTML}{DCE3FF}
\definecolor{fact_color_frame}{HTML}{3F69FF}
\definecolor{fact_color_title}{HTML}{FFFFFF}
\newtcbtheorem[number within=section]{fact}{Fact}
{
  skin=enhanced,
  boxrule = 0.75mm,
  sharp corners,
  % Title settings
  attach boxed title to top left = {
      yshift*=-\tcboxedtitleheight/2,
      xshift = 10mm,
    },
  boxed title style = {colback=fact_color_frame, sharp corners},
  % Adjust colors
  colback= fact_color_background,
  colframe= fact_color_frame,
  coltitle = fact_color_title,
}{fact}

% Problem environment
\definecolor{problem_color_background}{HTML}{FFDFCF}
\definecolor{problem_color_frame}{HTML}{FF5200}
\definecolor{problem_color_title}{HTML}{000000}
\newtcbtheorem[number within=section]{problem}{Problem}
{
  skin=enhanced,
  boxrule = 0.75mm,
  sharp corners,
  % Title settings
  attach boxed title to top left = {
      yshift*=-\tcboxedtitleheight/2,
      xshift = 10mm,
    },
  boxed title style = {colback=problem_color_frame, sharp corners},
  % Adjust colors
  colback= problem_color_background,
  colframe= problem_color_frame,
  coltitle = problem_color_title,
}{problem}








\usepackage{mathrsfs}


\begin{document}

\title{Proofs problems}
\author{Joshua Morton}
\maketitle

\chapter{Intuitive Proofs}


\begin{fact}{The pigeonhole principle}{PHP}
  \textbf{Simple form:} If $n + 1$ objects are placed into $n$ boxes, then at least one box has at least two objects in it. \\
  \textbf{General form:} If $kn + 1$ objects are placed into $n$ boxes, then at least one box has at least $k + 1$ objects in it.
\end{fact}

\begin{proposition*}{}
  If one chooses $n+1$ numbers from $\{1, 2, 3, \ldots, 2n\}$, it is guaranteed that two of the numbers they chose are consecutive.
\end{proposition*}

\begin{proof}
  TODO
\end{proof}

\begin{proposition*}{}
  If one selects any $n + 1$ numbers from the set $\{1, 2,\ldots,2n\}$, then two of the selected numbers will sum to $2n + 1.$
\end{proposition*}

\begin{proof}
  TODO
\end{proof}

\begin{proposition*}{}
  If one chooses 31 numbers from the set $\{1,2,3,\ldots,60\}$, then two of the numbers must be relatively prime.
\end{proposition*}

\begin{proof}
  TODO
\end{proof}

\begin{problem*}{}
  Determine whether or not the pigeonhole principle guarantees that two students at your school have the same 3-letter initals.
\end{problem*}

TODO

\chapter{Direct proofs}

\begin{fact}{}{IntegerClosure}
  The sum of integers in an integer, the difference of integers is an integer, and the product of integers is an integer.
\end{fact}

\begin{definition}{Even and odd integers}{EvenAndOdd}
  \begin{itemize}
    \item[$\bullet$] An integer $n$ is even if $n = 2k$ for some integer $k$;
    \item[$\bullet$] An integer $n$ is odd if $n = 2k + 1$ for some integer $k$.
  \end{itemize}

  Fact: Any integer is either even or odd.

\end{definition}

\begin{proposition*}{}
  The sum of an even integer and an odd integer is odd.
\end{proposition*}

\begin{proof}
  Assume that $n$ is an even integer and that $m$ is an odd integer.
  By the definition of even and odd numbers $n = 2a$ and $m = 2b + 1$ for some integers $a$ and $b$.
  Then, $$n + m = (2a) + (2b + 1) = 2a + 2b + 1 = 2(a + b) + 1.$$
  And since $a + b$ is an integer by Fact 2.0.1, we have shown that $n + m = 2k + 1$ where $k = a + b$.
  Therefore by the definition of an odd integer this means that $a + b$ is odd.
\end{proof}

\begin{proposition*}{}
  The product of two even integers is even.
\end{proposition*}

\begin{proof}
  Assume that $n$ and $m$ are even integers.
  By the definition of an even integer $n = 2a$ and $m = 2b$ for some integers $a$ and $b$.
  Then, $$nm = (2a)(2b) = 4ab = 2(2ab).$$
  And since $2ab$ is an integer by Fact 2.0.1, we have shown that $nm = 2k$ where $k = 2ab$.
  Therefore by the definition of an even integer this means that $nm$ is even.
\end{proof}

\begin{proposition*}{}
  The product of two odd integers is odd.
\end{proposition*}

\begin{proof}
  Assume that $n$ and $m$ are odd integers.
  By the definition of an odd integer this means that $n = 2a + 1$ and $m = 2b + 1$ for some integers $a$ and $b$.
  Then, $$nm = (2a + 1)(2b + 1) = 4ab + 2a + 2b + 1 = 2(2ab + a + b) + 1.$$
  And since $2ab + a + b$ is an integer by Fact 2.0.1, we have shown that $nm = 2k + 1$ where $k = 2ab + a + b$.
  Therefore by the definition of an odd integer this means that $nm$ is odd.
\end{proof}

\begin{proposition*}{}
  The product of an even integer and an odd integer is even.
\end{proposition*}

\begin{proof}
  Assume that $n$ is an even integer and $m$ is an odd integer.
  By the definition of an even and odd integer this means that $n = 2a$ and $m = 2b + 1$ for some integers $a$ and $b$.
  Then, $$nm = (2a)(2b + 1) = 4ab + 2a = 2(2ab + a).$$
  Since $2ab + a$ is an integer by Fact 2.0.1, we have shown that $nm = 2k$ where $k = 2ab + a$.
  Therefore by the definition of an even integer this means that $nm$ is even.
\end{proof}

\begin{proposition*}{}
  An even integer squared is an even integer.
\end{proposition*}

\begin{proof}
  Assume that $n$ is an even integer.
  By the definition of an even integer $n = 2a$ for some integer $a$.
  Then, $$n^2 = (2a)^2 = 4a^2 = 2(2a^2).$$
  Since $2a^2$ is an integer by Fact 2.0.1, we have shown that $n^2 = 2k$ where $k = 2a^2$.
  Therefore by the definition of an even integer this means that $n^2$ is even.
\end{proof}

\begin{definition}{}{}
  A nonzero integer $a$ is said to \emph{divide} an integer $b$ if $b = ak$ for some integer $k$.
  When $a$ does divide $b$, we write $``a \mid b"$ and when $a$ does not divide $b$ we write $``a \nmid b."$
\end{definition}

\begin{proposition}{}{}
  If $d \mid a$ and $d \mid b$ then $d \mid a + b$.
\end{proposition}

\begin{proof}
  Assume that $d \mid a$ and $d \mid b$.
  By the definition of divisibility $a = dk$ and $b = dl$ for some integers $k$ and $l$.
  Then, $$a + b = dk + dl = d(k + l).$$
  Since $k + l$ is an integer by Fact 2.0.1, we have shown that $a + b = dq$ where $q = k + l$.
  Therefore by the definition of divisibility this means that $d \mid a + b$.
\end{proof}

\begin{proposition}{}{}
  If $d \mid b$ then $d \mid -b$.
\end{proposition}

\begin{proof}
  Assume that $d \mid b$.
  By the definition of divisibility $dk = b$ for some integer $k$.
  Then, $$-b = -(dk) = d(-k).$$
  Since $-k$ is an integer by Fact 2.0.1, we have shown that $-b = dq$ where $q = -k$.
  Therefore by the definition of divisibility this means that $d \mid -b$.
\end{proof}

\begin{proposition}{}{}
  If $d \mid b$ then $-d \mid b$.
\end{proposition}

\begin{proof}
  Assume that $d \mid b$.
  By the definition of divisibility $dk = b$ for some integer $k$.
  Then, $$b = dk = --dk = -d(-k)$$
  Since $-k$ is an integer by Fact 2.0.1, we have shown that $b = -dq$ where $q = -k$.
  Therefore by the definition of divisibility this means that $-d \mid b$.
\end{proof}

\begin{definition}{Modular Congruence}{}
  For integers $a, r$ and $m$, we say that $a$ is \emph{congruent} to $r$ \emph{modulo} $m$, and we write $a \equiv r \pmod{m}$, if $m \mid (a - r)$.
\end{definition}

\begin{proposition}{}{}
  If $a,b,c,d$ and $m$ are integers, $a \equiv b \pmod{m}$ and $c \equiv d \pmod{m}$. Then, $a + c \equiv b + d \pmod{m}$.
\end{proposition}

\begin{proof}
  Assume that $a,b,c,d$ and $m$ are integers, $a \equiv b \pmod{m}$ and $c \equiv d \pmod{m}$.
  By the definition of modular congruence this means that $m \mid (a - b)$ and $m \mid (c - d)$.
  Applying the definition of divisibility we get $mk = a - b$ and $ml = c - d$ for some integers $k$ and $l$.
  Then, $$(a + c) - (b + d) = (a - b) + (c - d) = mk + ml = m(k + l).$$
  Since by Fact 2.0.1 $k + l$ is an integer, we have shown that $(a + c) - (b + d) = mq$ where $q = k + l$.
  Therefore by the definition of divisibility $m \mid (a + c) - (b + d)$.
  Furthermore by the definition of modular congruence $a + c \equiv b + d \pmod{m}$.
\end{proof}

\begin{proposition}{}{}
  If $a,b,c,d$ and $m$ are integers, $a \equiv b \pmod{m}$ and $c \equiv d \pmod{m}$. Then, $a - c \equiv b - d \pmod{m}$.
\end{proposition}

\begin{proof}
  Assume that $a,b,c,d$ and $m$ are integers, $a \equiv b \pmod{m}$ and $c \equiv d \pmod{m}$.
  By the definition of modular congruence this means that $m \mid (a - b)$ and $m \mid (c - d)$.
  Applying the definition of divisibility we get $mk = a - b$ and $ml = c - d$ for some integers $k$ and $l$.
  Then, $$(a - c) -(b - d) = (a - b) -(c - d) = mk - ml = m(k-l).$$
  Since by Fact 2.0.1 $k - l$ is an integer, we have shown that $(a - c) - (b - d) = mq$ where $q = k - l$.
  Therefore by the definition of divisibility $m \mid (a - c) - (b - d)$.
  Furthermore by the definition of modular congruence $a - c \equiv b - d \pmod{m}$.
\end{proof}

\begin{proposition}{}{}
  If $a,b,c,d$ and $m$ are integers, $a \equiv b \pmod{m}$ and $c \equiv d \pmod{m}$. then, $ac \equiv bd \pmod{m}$.
\end{proposition}

\begin{proof}
  Assume that $a,b,c,d$ and $m$ are integers, $a \equiv b \pmod{m}$ and $c \equiv d \pmod{m}$.
  By the definition of modular congruence this means that $m \mid (a - b)$ and $m \mid (c - d)$.
  Applying the definition of divisibility we get $mk = a - b$ and $ml = c - d$ for some integers $k$ and $l$.
  Then, $$ac - bd = $$
  TODO
\end{proof}

\begin{definition}{Greatest common divisor}{}
  Given two integers $a$ and $b$, the \emph{greatest common divisor} of $a$ and $b$ is the largest integer $d$, such that $d \mid a$ and $d \mid b$.
  We say that the $\gcd(a, b) = d$.
\end{definition}

\begin{lemma}{}{}
  If $a$, $b$ and $c$ are integers and $\gcd(a, b)  = 1$ and $\gcd(a, c) = 1$ then, $\gcd(a, bc) = 1$.
\end{lemma}

\begin{proof}
  TODO
\end{proof}

\chapter{Sets}


\begin{definition}{Subsets}{}
  Suppose $A$ and $B$ are sets. If every element in $A$ is also in $B$, then $A$ is a \emph{subset} of $B$, denoted $A \subseteq B$.
\end{definition}

\begin{definition}{Union}{}
  The \emph{union} of sets $A$ and $B$ is the set $A \cup B = \{x : x \in A \lor x \in B\}$.
  Furthermore if $\mathscr{A}$ is a set of sets, then $\bigcup_{S \in \mathscr{A}} S$ is the \emph{union} between all subsets of $\mathscr{A}$.
\end{definition}

\begin{definition}{Intersection}{}
  The \emph{intersection} of sets $A$ and $B$ is the set $A \cap B = \{x : x \in A \land x \in B\}$.
  Furthermore if $\mathscr{A}$ is a set of sets, then $\bigcap_{S \in \mathscr{A}} S$ is the \emph{intersection} between all subsets of $\mathscr{A}$.
\end{definition}

\begin{definition}{Set subtraction}{}
  The \emph{subtraction} of sets $B$ from $A$ is the set $A \setminus B = \{x : x \in A \land x \notin B\}$.
\end{definition}

\begin{definition}{Complement of a set}{}
  If $A \subseteq U$, then $U$ is called a \emph{universal set} of $A$.
  The \emph{complement} of $A$ in $U$ is $A^c = U \setminus A$.

\end{definition}

\end{document}